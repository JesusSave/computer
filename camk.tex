\chapter{CAMK}
\section{Connect to Networks}
\begin{itemize}
\item\cmd{ssh chen@ssh.camk.edu.pl} connect to camk server
\item\cmd{ssh -X -J chen@ssh.camk.edu.pl chen@chen} connect wo camk server and connect to machine 'chen'
\item\cmd{ssh chuck} when connected with camk server, this connect further to cluster chuck
\end{itemize}

\subsection{Remote desktop}
\begin{enumerate}
\item The admian Staszek installed \cmd{vnc} on the computer 'chen'
\item \cmd{ssh chen@ssh.camk.edu.pl chen@chen} connect to computer 'chen'
\item \cmd{vncpasswd} to set the password for vnc-viewer. 'b*****g'
\item \cmd{sudo apt install tigervnc-viewer} install vnc-viewer on the remote computer
\item \cmd{vncviewer -via chen@ssh.camk.edu.pl chen} remotely connect to the computer desktop 'chen'
\end{enumerate}

\subsection{Laptop}
Laptop to connect to networks: \cmd{ssh -X chen@ssh.camk.edu.pl}, optional -X or -Y flag enable client X11 open figures in the server

\subsection{Office Desktop}
Office computer connection: \cmd{ssh chen@chen}\\
The first chen is my camk account, the second chen is my computer name.

\section{Printer}
\begin{itemize}
\item \cmd{73691945} To jest mój password. lp3c(only scan), lp4c (new printer), lp5c (no need of passward)
\item \cmd{/scratch/1m/chen/lp3-scanner} is the storage folder of my scanned material
\end{itemize}

\section{Storage}
\begin{enumerate}
\item\cmd{/work/archive/Lectures} lectures video recorded.
\end{enumerate}

\section{Cluster Chuck and SLURM}
Formerly the cluster name is PSK. From 2018, CAMK has new cluster chuck, using SLURM to allocate jobs

\begin{itemize}
\itemsep0em
\item \cmd{ssh chuck} login
\item \cmd{sinfo --help} read tips 
\item \cmd{sinfo -N -l} node-oriented fashion, and more informatoin
\item \cmd{sbatch example.sh} submit the job
\item \cmd{squeue} see current job queue
\item \cmd{squeue -u USER} see jobs of specific user
\item \cmd{scontrol show job JOBID} check detail of the job
\item \cmd{scancel JOBID} kill the job
\item Job States: PD pending, R running, CD completed, CA canceled, F failed
\item MPI jobs
\end{itemize}


\subsection{Disk quota}
\begin{itemize}
\item \cmd{quota} check my disk quota
\item \cmd{quota -vs} check my disk quota in human readable unit
\item \cmd{quota -u USER} check disk quota
\end{itemize}

\subsection{Time limit}
\begin{itemize}
\item \cmd{sacct -e} check elements list
\item \cmd{sacct --format="Timelimit"} check time limit of runing job
\item \cmd{scontrol update jobid=<JOBID> TimeLimit=<newtimelimit>} adding time to running job, requires admin privileges on some machines. CAMK max is 7 days.
\end{itemize}

\section{chuck example with Zeltron code}
\subsection{Log in}
\begin{itemize}
\item \cmd{ssh chen@chuck} login the CAMK local cluster chuck
\end{itemize}

\subsection{Setting simulation configuration and SLURM job bash script}

\begin{itemize}
\item \cmd{vi mod\_input.f90} setting simulation parameters
\item \cmd{vi submit\_zeltron\_chuck.sh} edit the SLURM bash script before submitting the job
  \begin{itemize}
  \item time limit, chuck upper limit is 7 days
  \item data storaging path
  \item CPUs and nodes, the number of processors n should be the same as NPX*NPY in Zetron setting by mod\_input.f90. In Harris layer simulation, 1) Better to set NPY number as $4*$ times to allocate particles equally. 2) Better to increase NPX than NPY because anisotropic configuration
  \item Number of nodes (N), one node in chuck has $N = 20$ processors.
  \end{itemize}
\end{itemize}

\subsection{Submiting and checking the job}

\begin{itemize}
\item \cmd{ssh chuck} log into chuck cluster
\item \cmd{cd /work/chuck/chen/harris01} Go to the work folder
\item \cmd{./run\_zeltron.sh} compile and submitting the job into the queue. Or manually as
  \begin{itemize}
  \item \cmd{module load mpi} load the Message Passing Interface (MPI) module
  \item \cmd{make} compile zeltron code 
  \item \cmd{sbatch submit\_zeltron\_chuck.s} submit the job into the queue
  \end{itemize}
\item \cmd{squeue -u chen} check the job status
\end{itemize}

\subsection{Results Analysis}


Quick plot in the server

\begin{itemize}
\item \cmd{gnuplot} start gnuplot to quick plot the map of magnetic field in the server
\item \cmd{gnuplot> plot 'spec_ele_t1000.dat'} plot 1D data
\item \cmd{plot "data.txt" using 1:2} \red{plot coloum 2 on 1}
\item \cmd{gnuplot> plot 'Bx\_t0.dat' matrix w image} plot map
\end{itemize}


Check in local worksation

\begin{itemize}
\item\cmd{python diagnose.py} (this does not work on chuck, can be called at local workstation)
\item parameters.dat shows some basic parameter values
\item zeltron.log shows the current status of the program
\item Eem.dat and similar files contain information on the total energy content, can be plotted with
\end{itemize}


\subsection{Errors}

\begin{itemize}
\item mpirun signal 9 (killed) problem: Memory overflow issue. Memery is per CPU 'SBATCH --mem-per-cpu=3GB', so CUP number is important.
\end{itemize}



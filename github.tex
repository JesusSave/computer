\chapter{GitHub}

\begin{chapquote}{Proverbs 4:7, \textit{ASV Bible}}
``Wisdom is supreme. Get wisdom. Yes, though it costs all your possessions, get understanding.''
\end{chapquote}

\section{About}
\begin{enumerate}
\item Linus created Git in 2005 for Linux development. 
\item Git是分佈式版本控制系統 
\end{enumerate}

\section{Create reposites}
\begin{enumerate}
\item\cmd{git init} Initial folder as 工作區
\item\cmd{git add filename} add filename to the 暫存區 stage (index) in \verb|.git|  
\item\cmd{git add .} Add all files to 暫存區.
\item\cmd{git commit -m "description for this commit"} add description to this commit, 暫存區內容提交到 master (head) 分支
\item\cmd{git push origin master} push file to github.
\end{enumerate}

\section{Versions}
\subsection{Files}
\begin{enumerate}
\item\cmd{git status} check status
\item\cmd{git diff filename} check difference of filename
\item\cmd{git checkout -- filename} 撤銷filename 到最後的狀態
\item\cmd{git reset HEAD filename} 把暫存區的修改撤銷掉(unstage),重新放回工作區
\item\cmd{git rm filename} remove file in the 暫存區, and commit it
\end{enumerate}

\subsection{Reposits}
\begin{enumerate}
\item\cmd{git log} check 3 recent history entry
\item\cmd{git reset --hard HEAD^} retreat to last version 
\item\cmd{git reset --hard commitprefix} reset head version according to coomitprefix (few digits)
\item\cmd{git reflog} check commands log
\end{enumerate}

\section{Upload}
\subsection{SSH Key}
\begin{enumerate}
\item\cmd{ssh-keygen -t rsa -C "c275633094@gmal.com"} create SSH Key, password can be none. \verb|.ssh/id_rsa| 爲私鑰,\verb|./ssh/id_rsa.pub| 是公鑰
\item GitHub> Account settings> SSH Keys> Add SSH Key> paste contents in \verb|id_rsa.pub|, 每個電腦單獨生成公鑰
\end{enumerate}

\subsection{Create a new repo}
\begin{enumerate}
\item GitHub>Create a new repo>Create
\item\cmd{git remote add origin git@github.com:JesusSave/mGRB_afterglow.git} 關聯本地與遠程的庫 
\item\cmd{git push -u origin master} \verb|-u| 把本地和遠程的master 分支關聯起來,一次之後就可以簡化這個標籤
\item\cmd{git remote -v} 查看遠程庫信息
\item\cmd{git remote rm origin} 刪除本地和遠程的綁定關係
\end{enumerate}

\subsection{Clone a reposite}
\begin{enumerate}
\item\cmd{git clone git@github.com:JesusSave/mGRB_afterglow.git} 克隆遠程庫
\end{enumerate}

\section{Brance}
\begin{enumerate}
\item\cmd{git switch -c dev} switch to new created branch 'dev'
\item\cmd{git branch} check current branch
\item\cmd{git switch master} switching to branch 'master'
\item\cmd{git merge dev} merge branch dev to current branch
\item\cmd{git merge --no-ff dev} merge without fast forward, 別人看不出合併信息
\item\cmd{git branch -d dev} delete branch 'dev
\item 實際開發中,master 用於發布新版本,dev 用於平時開發,不同開發人員合並到 dev 上
\end{enumerate}

\subsection{Bug}
\begin{enumerate}
\item\cmd{git stash} 把當前工作現場儲藏起來 
\item\cmd{git stash list} 工作現場列表
\item\cmd{git stash apply} 恢復工作現場
\item\cmd{git stash drop} 刪除儲藏的工作現場
\item\cmd{git stash pop} 恢復並同時把stash 內容刪除
\item\cmd{git cherry-pick 4c805e2} 複製 4c805e2 提交的變化到當前分支
\end{enumerate}

\subsection{Feature}
\begin{enumerate}
\item 開發新功能,爲避免混亂,建議另建立一個分支,名爲 feature 
\item\cmd{git branch -D feature-vulcan} 強行刪除 feature-vulcan 分支
\end{enumerate}

\subsection{Pull}
\begin{enumerate}
\item\cmd{git branch --set-upstream-to=origin/dev dev} 將遠程origin/dev 與本地 dev 連接起來
\item\cmd{git pull} 抓遠程到本地 
\end{enumerate}

\section{Tag}
\begin{enumerate}
\item\cmd{git tag v1.0} 加入新標籤v1.0
\item\cmd{git tag v0.9 f52c633} 對某次commit id 打標籤
\item\cmd{git tag -a v0.9 -m "version 0.9 released" f52c633} 創建帶說明的標籤
\item\cmd{git tag} 查看所有標籤
\item\cmd{git push origin v1.0} 推送標籤到遠程
\item\cmd{git push origin --tags} 推送所有標籤到遠程
\item\cmd{git tag -d v0.9} 刪除本地標籤
\item\cmd{git push origin :refs/tags/v0.9} 刪除遠程標籤
\end{enumerate}

\chapter{Package}

\section{Useful Apps}
\begin{table}[ht]
\caption{App lists}
\centering
\begin{tabular}{c c c c}
\hline\hline
Name & example & notes \\ [0.5ex]
\hline
wc & {wc 1.dat} & word count of line, word, bytes\\[1ex]
paste & paste -sd+ timestep.dat | bc &  print the sum of one column data\\
sed & {sed -i.bak -e '5,10d;12d' file} & delete 5 to 10 and 12 line.\\
expr & {expr 14 \% 9} & 整数运算 14-9 = 5 \\
expr & {expr 14 - 9} & 整数运算 14-9 =  \\
cat & {cat 3.dat >> 1.dat} & combine 2.dat to 1.dat.\\
\hline
\end{tabular}
\label{tab_apps}
\end{table}

\section{Install and Remove}
\begin{itemize}
\item\cmd{apt-cache showpkg appname} show available package versions
\item\cmd{apt-mark hold appname} on hold app prevent upgrading
\item\cmd{apt-cache rdepends packagename} show package dependance
\item\cmd{sudo apt-get install <package-name>=<package-version-number>} install according to version
\item\cmd{sudo apt install appname}
\item\cmd{sudo apt remove appname}
\item\cmd{sudo apt-get purge nvidia-*} purge nvidia
\item\cmd{sudo apt-get autoclean} to clean up partial pakages
\item\cmd{sudo apt-get autoremove} to clean up apt cache
\item\cmd{sudo apt-get clean} to remove any unused dependencies
\item\cmd{man apt-get} to get more info on apt-get and how to use it.
\item\cmd{/var/log/apt/history.log} is the history of apt
\end{itemize}

\subsection{Upgrade upgradable}
\begin{itemize}
\item\cmd{apt list --upgradable}
\item\cmd{sudo apt-get dist-upgrade} upgrade upgradable with denpendencies
\end{itemize}

\section{Install from local file}
\cmd{sudo dpkg -i example.deb} install \\
\cmd{sudo dpkg -r linuxqq} Remove

\begin{lstlisting}[language=bash, caption={Purge Xfce}]
$dpkg -l | grep .xfce. | awk '{print $2}' |
xargs sudo apt-get purge -V --auto-remove -yy
\end{lstlisting}

\section{Check installed}
\begin{itemize}
\item \cmd{dpkg -s mplayer} check if mplayer is installed
\item \cmd{which mplayer} check the path of mplayer 
\end{itemize}


\section{Personal Package Archive (PPA)}

\begin{itemize}
\item\cmd{/etc/apt/sources.list} is the main PPA list
\item\cmd{/etc/apt/sources.list.d/} is the folder for personal PPA list
\item\cmd{sudo apt-add-repository ppa:whatever/ppa} install ppa
\item\cmd{sudo add-apt-repository --remove ppa:whatever/ppa} remove ppa
\item Alternative,
\cmd{ls /etc/apt/sources.list.d}
\item \cmd{sudo rm -i /etc/apt/sources.list.d/myppa.list}
\end{itemize}

\subsection{public key}

\begin{itemize}
\item if \cmd{Err:1 http://ftp.agh.edu.pl/ubuntu bionic-updates InRelease                  
  The following signatures couldn't be verified because the public key is not available: NO_PUBKEY 3B4FE6ACC0B21F32}
\item then \cmd{sudo apt-key adv --keyserver keyserver.ubuntu.com --recv-keys 3B4FE6ACC0B21F32}
\end{itemize}


\subsection{Default Repository}

\begin{itemize}
\item \cmd{sudo mv /etc/apt/sources.list ~} backup your current source list
\item \cmd{sudo touch /etc/apt/sources.list} create an empty list
\item \cmd{software-properties-gtk} Open Software \& Updates, and choose the canonical source and main server
\end{itemize}


\section{Packages Broken}
\begin{enumerate}
\item\cmd{Errors were encountered while processing: /var/cache/apt/archives/libglx-mesa0_18.0.5-0ubuntu0~18.04.1_amd64.deb}\\
when errors happen, then
\item\cmd{sudo apt --force-overwrite /var/cache/apt/archives/libglx-mesa0_18.0.5-0ubuntu0~18.04.1_amd64.deb} \\
force install when have depended packages
\item\cmd{dpkg -P --force-depends App} to remove depended packages
\end{enumerate}

\subsection{apt --fix-broken install}
\begin{enumerate}
\item\cmd{sudo apt -o Dpkg::Options::="--force-overwrite" --fix-broken install}\footnote{\url{https://unix.stackexchange.com/a/624842/266769}}
\end{enumerate}

\section{Install without sudo}

\begin{lstlisting}[language=bash, caption={dpkg install from deb}]
apt download appname   # download the package
dpkg -x package.deb dir # install to dir
\end{lstlisting}


\begin{lstlisting}[language=bash, caption={manually install from deb}]
cd ~/chen_install
apt download mupdf # dowmlad mupdf package
ar x mupdf*.deb 
tar xvf data.tar.gz
PATH="$PATH":~/chen_install/usr/bin
\end{lstlisting}

\begin{lstlisting}[language=bash, caption={run appimage}]
chmod a+x example.AppImage
./example.AppImage
\end{lstlisting}



\subsection{install from source}

\begin{lstlisting}[language=bash, caption={install from source}]
apt-get source package # doesn't work???
cd package
./configure --prefix=$HOME/chen_install # install to chen_install
make
make install
\end{lstlisting}


\section{PATH}

\begin{itemize}
\item Shell PATH 变量用于系统查找命令的路径
\item \verb|echo $PATH| 查询当前 PATH 环境变量
\item \verb|cp my_app.sh /bin/| 可以把自己的脚本复制到 PATH 变量定义的路径(比如/bin/),不用自己再写PATH
\item \verb|PATH="$PATH":~/chen_install/usr/bin| 通过变量叠加的方式临时加入我自己的路径,注销后失效
\item \verb|~/.bash_aliases| is a sub file of \verb|~/.bashrc|, and is dedicated for the client to define PATH
\end{itemize}


\subsection{temperary client}

\begin{itemize}
\item 以添加mongodb server为列
\item \verb|export PATH=/usr/local/mongodb/bin:$PATH|
\item 生效方法:立即生效
\item 有效期限:临时改变,只能在当前的终端窗口中有效,当前窗口关闭后就会恢复原有的path配置
\item 用户局限:仅对当前用户
\end{itemize}
 
\subsection{.bashrc permanent client}

\begin{itemize}
\item \cmd{vim ~/.bashrc}
\item 在最后一行添上: \cmd{export PATH=/usr/local/mongodb/bin:$PATH}
\item 生效方法:
  \begin{itemize}
  \item 1、对新终端窗口生效
  \item 2、之前打开的窗口可刷新\cmd{source ~/.bashrc} 生效
  \end{itemize}
\item 有效期限:永久有效
\item 用户局限:仅对当前用户
\end{itemize}

\subsection{profile All clinets permanant}

\begin{itemize}
\item \cmd{vim /etc/profile}
\item 找到设置PATH的行,添加\cmd{export PATH=/usr/local/mongodb/bin:$PATH}
\item 生效方法:系统重启
\item 有效期限:永久有效
\item 用户局限:对所有用户
\end{itemize}
 
\subsection{environment permanent}

\begin{itemize}
\item \cmd{vim /etc/environment}
\item 原路径 \cmd{PATH="/usr/local/sbin:/usr/local/bin:..."}
\item 在其后添加新路径 \cmd{":/usr/local/mongodb/bin"}
\item 生效方法:系统重启
\item 有效期限:永久有效
\item \red{用户局限如何?}
\end{itemize}


\section{SVN}

\begin{lstlisting}[language=bash, caption={Download source files of PDFsandwich}]
svn checkout svn://svn.code.sf.net/p/pdfsandwich/code/trunk/src pdfsandwich
\end{lstlisting}

\chapter{Network}
\section{Basic}
\begin{enumerate}
\item LAN: local area network 局域網
\item ping: ping(呯)是一种计算机网络工具,用來測試数据包能否透過IP协议到達特定主機。因為這個程式的運作原理与潛水艇的主动声纳相似,他便用聲納的聲音來為程式取名。网络管理员之间也常将ping用作动词,如“ping一下计算机XXX,看它是否开着。”
\item DNS: domain name system, 網域名稱系統,是互聯網的一項服務。它作为将域名和IP地址相互映射的一个分布式数据库,能够使人更方便地访问互联网。
\end{enumerate}

\section{IP}
\begin{enumerate}
\item \cmd{hostname -I} check IP
\item \cmd{ip addr} show IP informations
\item \cmd{ip addr | grep eth0} show eth0 IP information
\item \cmd{ping 172.25.32.1} ping my IFPiLM desktop win11 IP to check if open
\end{enumerate}

\section{SSH}
\begin{enumerate}
\item SSH (secure shell) 是一种加密的网络传输协议,可在不安全的网络中为网络服务提供安全的传输环境.
\item\cmd{sudo systemctl restart ssh.serivce}
\end{enumerate}

\subsection{Make Linux as server in the local area network}
\begin{enumerate}
\item\cmd{sudo apt install openssh-server} install openssh-server\footnote{\url{https://blog.csdn.net/Xiao_DANDAN110/article/details/115385088}}
\item\cmd{ssh localhost} check if SSH is installed
\item\cmd{ps -e | grep ssh} check SSH services started or not
\item\cmd{ifconig} check IP (in wlp3s0, e.g. \verb|10.0.0.140|)
\begin{lstlisting}[language=bash, caption={ifconig (wi-fi connected)}]
chen@4-726:~$ ifconfig 
enp2s0: flags=4099<UP,BROADCAST,MULTICAST>  mtu 1500
        ether 10:7d:1a:47:a8:b9  txqueuelen 1000  (Ethernet)
        RX packets 0  bytes 0 (0.0 B)
        RX errors 0  dropped 0  overruns 0  frame 0
        TX packets 0  bytes 0 (0.0 B)
        TX errors 0  dropped 0 overruns 0  carrier 0  collisions 0

lo: flags=73<UP,LOOPBACK,RUNNING>  mtu 65536
        inet 127.0.0.1  netmask 255.0.0.0
        inet6 ::1  prefixlen 128  scopeid 0x10<host>
        loop  txqueuelen 1000  (Local Loopback)
        RX packets 5313867  bytes 28095323793 (28.0 GB)
        RX errors 0  dropped 0  overruns 0  frame 0
        TX packets 5313867  bytes 28095323793 (28.0 GB)
        TX errors 0  dropped 0 overruns 0  carrier 0  collisions 0

wlp3s0: flags=4163<UP,BROADCAST,RUNNING,MULTICAST>  mtu 1500
        inet 10.0.0.140  netmask 255.255.255.0  broadcast 10.0.0.255
        inet6 fe80::e482:8042:b2f1:8724  prefixlen 64  scopeid 0x20<link>
        ether d4:6a:6a:65:33:85  txqueuelen 1000  (Ethernet)
        RX packets 20568393  bytes 29189561323 (29.1 GB)
        RX errors 0  dropped 48614  overruns 0  frame 0
        TX packets 2325362  bytes 217116170 (217.1 MB)
        TX errors 0  dropped 0 overruns 0  carrier 0  collisions 0
\end{lstlisting}
\begin{lstlisting}[language=bash, caption={ifconig (wired connected)}]
chen@4-726:~$ ifconfig
enp2s0: flags=4163<UP,BROADCAST,RUNNING,MULTICAST>  mtu 1500
        inet 10.0.0.140  netmask 255.255.255.0  broadcast 10.0.0.255
        inet6 fe80::e88a:c3c0:d7b1:60bf  prefixlen 64  scopeid 0x20<link>
        ether 10:7d:1a:47:a8:b9  txqueuelen 1000  (Ethernet)
        RX packets 1103501  bytes 1332761045 (1.3 GB)
        RX errors 0  dropped 26680  overruns 0  frame 0
        TX packets 310128  bytes 77010654 (77.0 MB)
        TX errors 0  dropped 0 overruns 0  carrier 0  collisions 0

lo: flags=73<UP,LOOPBACK,RUNNING>  mtu 65536
        inet 127.0.0.1  netmask 255.0.0.0
        inet6 ::1  prefixlen 128  scopeid 0x10<host>
        loop  txqueuelen 1000  (Local Loopback)
        RX packets 5334803  bytes 28203353224 (28.2 GB)
        RX errors 0  dropped 0  overruns 0  frame 0
        TX packets 5334803  bytes 28203353224 (28.2 GB)
        TX errors 0  dropped 0 overruns 0  carrier 0  collisions 0

wlp3s0: flags=4163<UP,BROADCAST,RUNNING,MULTICAST>  mtu 1500
        inet 10.42.0.1  netmask 255.255.255.0  broadcast 10.42.0.255
        inet6 fe80::2957:3940:18bd:8b63  prefixlen 64  scopeid 0x20<link>
        ether d4:6a:6a:65:33:85  txqueuelen 1000  (Ethernet)
        RX packets 20926168  bytes 29324993951 (29.3 GB)
        RX errors 0  dropped 84130  overruns 0  frame 0
        TX packets 3249623  bytes 1446235048 (1.4 GB)
        TX errors 0  dropped 0 overruns 0  carrier 0  collisions 0

\end{lstlisting}
\item other local area network computers can visit above host by \cmd{ssh chen@10.0.0.140} (enp2s0 for wired and wlp3s0 for wi-fi both have this IP)
\end{enumerate}

\subsection{screen}

\cmd{screen} installed in server, in ssh to initilize multipul terminal

\cmd{ctrl+a, |} split vertically

\cmd{ctrl+a, S} split horizontally

\cmd{ctrl+a, Q} unsplit

\cmd{ctrl+a, tab} switch terminal

\cmd{ctrl+a, c} to use new region

\cmd{ctrl+a, space} next terminal

\cmd{ctrl+a, backspace}, previous terminal

\cmd{ctrl+a, number} choose terminal

\cmd{ctrl+a, "} choose terminal

\cmd{ctrl+a, a} to the underlying terminal

\section{hosts}

\cmd{/etc/hosts}

\cmd{hostname} show your name

\cmd{hostid}

\section{Network}

\begin{itemize}
\item\cmd{nmcli connection show $connection_uuid} show the connections
\item\cmd{nmcli connection modify BlackBerry\ BBB100-2\ 2219\ 1 connection.metered no} set NAME of the connection as not metered
\end{itemize}

\section{WIFI}
\begin{enumerate}
\item\cmd{nm-connection-editor} the Wifi connection lists
\item\cmd{sudo vim /etc/NetworkManager/system-connections/somename.nmconnection} edit connection 'somename'
\item\cmd{autoconnect-priority=10} higher number means higher priority, can negative the unwanted connection
\end{enumerate}

\subsection{Eduroam}
\begin{lstlisting}[language=bash, caption={/home/chen/.ssh/eduroam.txt}]
## This is your eduroam credentials
Username: chen@eduroam.camk.edu.pl
Password: hHvZEFlqYDiEYXK2
Certificate on https://eduroam.camk.edu.pl/
## You can also use https://cat.eduroam.org/, search for CAMK
\end{lstlisting}

\subsection{Passwords memo}
\begin{itemize}
\item\cmd{sudo vim /etc/NetworkManager/system-connections/somename.nmconnection} check already restored passwords
\item \cmd{CAMK} wifi password \cmd{a w sercu maj}
\item \cmd{ifpilm-wlan} wifi password \cmd{B**.}
\end{itemize}

\subsection{WIFI card 無線網卡}
\begin{enumerate}
\item managed mode: 被管理模式,作爲客戶端,與 Access Point (AP) 相聯,比如筆記本的WIFI卡與路由器相聯 
\item master mode: 作爲 AP,即分享熱點
\item ad hoc mode: 對等模式,兩個設備互聯
\item monitor mode: 監聽模式,監聽無線網內部流量。比如我猜,筆記本作爲監聽者,瞭解手機和路由器之間的流量。
\end{enumerate}

\subsection{TP-link TL-WN722N V3 Monitor Mode}
\begin{enumerate}
\item\cmd{lsusb} show usb
\begin{lstlisting}[language=bash, caption={lsusb}]
Bus 001 Device 002: ID 2357:010c TP-Link TL-WN722N v2/v3 [Realtek RTL8188EUS]
\end{lstlisting}
\item\cmd{iwconfig} show wireless networks
\begin{lstlisting}[language=bash, caption={iwconfig}]
wlx503eaa6e7140  unassociated  ESSID:""  Nickname:"<WIFI@REALTEK>"
          Mode:Managed  Frequency=2.412 GHz  Access Point: Not-Associated   
          Sensitivity:0/0  
          Retry:off   RTS thr:off   Fragment thr:off
          Power Management:off
          Link Quality=0/100  Signal level=0 dBm  Noise level=0 dBm
          Rx invalid nwid:0  Rx invalid crypt:0  Rx invalid frag:0
          Tx excessive retries:0  Invalid misc:0   Missed beacon:0
\end{lstlisting}
\item\cmd{sudo iwconfig wlx503eaa6e7140 mode monitor}
\begin{lstlisting}[language=bash, caption={sudo iwconfig wlx503eaa6e7140 mode monitor}]
Error for wireless request "Set Mode" (8B06) :                                                                
    SET failed on device wlx503eaa6e7140 ; Invalid argument.
\end{lstlisting}
\item Install driver\footnote{\url{https://www.hackster.io/thatiotguy/enable-monitor-mode-in-tp-link-tl-wn722n-v2-v3-128fc6}}
\begin{lstlisting}[language=bash, caption={install driver rtl8188eus}]
sudo apt install bc
sudo rmmod r8188eu.ko # rm module from kernel
git clone https://github.com/aircrack-ng/rtl8188eus
cd rtl8188eus
sudo -i
echo "blacklist r8188eu" > "/etc/modprobe.d/realtek.conf"
exit
make
sudo make install
sudo modprobe 8188eu #add module to kernel
\end{lstlisting}
\item\cmd{iwconfig wlx503eaa6e7140 mode monitor} change mode to monitor
\end{enumerate}

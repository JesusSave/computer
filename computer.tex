% -*- coding: utf-8 -*-
%%%%%%%%%%%%%%%%%%%%%%%%%%%%%%%%%%%%%%%
% Notebook of astrophysics
% Version 1.1 (17/3/2017)
% Author : Yiqi Xinzhu
% c275633094@gmail.com
% 注意: Windows下中文编码用GBK,Linux编码为Utf8,否则乱码哟!一起信主20170228 
%%%%%%%%%%%%%%%%%%%%%%%%%%%%%%%%%%%%%%

\documentclass[a4paper,11pt]{book}
\usepackage[T1]{fontenc}
\usepackage[utf8]{inputenc}
\usepackage{lmodern}
%%%%%%%%%%%%%%%%%%%%%%%%%%%%%%%%%%%%%%%%%%%%%%%%%%%%%%%%%
% Source: http://en.wikibooks.org/wiki/LaTeX/Hyperlinks %
%%%%%%%%%%%%%%%%%%%%%%%%%%%%%%%%%%%%%%%%%%%%%%%%%%%%%%%%%
\usepackage{CJKutf8} % Chinese coding package
\usepackage[unicode={true}]{hyperref} % 最好保证 hyperref 是最后加载的宏包,The hy­per­ref pack­age is used to han­dle cross-ref­er­enc­ing com­mands in LATEX to pro­duce hy­per­text links in the doc­u­ment.
\usepackage{graphicx}
\usepackage{amsmath}
\usepackage{empheq} % Boxes
%\usepackage[english]{babel}
\usepackage{tikz}
\usepackage{pgfplots} % axis
\usetikzlibrary{arrows.meta} % arrow style 
\usetikzlibrary{quotes,angles}
\usetikzlibrary{calc,decorations.pathmorphing}
\usetikzlibrary{fit}
\usetikzlibrary{positioning}

%%%%%%%%%%%%%%%%%%%%%%%%%%%%%%%%%%%%%%%%%%%%%%%%%%%%%%%%%%%%%%%%%%%%%%%%%%%%%%%%
% 'dedication' environment: To add a dedication paragraph at the start of book %
% Source: http://www.tug.org/pipermail/texhax/2010-June/015184.html            %
%%%%%%%%%%%%%%%%%%%%%%%%%%%%%%%%%%%%%%%%%%%%%%%%%%%%%%%%%%%%%%%%%%%%%%%%%%%%%%%%
\newenvironment{dedication}
{
   \cleardoublepage
   \thispagestyle{empty}
   \vspace*{\stretch{1}}
   \hfill\begin{minipage}[t]{0.66\textwidth}
   \raggedright
}
{
   \end{minipage}
   \vspace*{\stretch{3}}
   \clearpage
}

%%%%%%%%%%%%%%%%%%%%%%%%%%%%%%%%%%%%%%%%%%%%%%%%
% Chapter quote at the start of chapter        %
% Source: http://tex.stackexchange.com/a/53380 %
%%%%%%%%%%%%%%%%%%%%%%%%%%%%%%%%%%%%%%%%%%%%%%%%
\makeatletter
\renewcommand{\@chapapp}{}% Not necessary...
\newenvironment{chapquote}[2][2em]
  {\setlength{\@tempdima}{#1}%
   \def\chapquote@author{#2}%
   \parshape 1 \@tempdima \dimexpr\textwidth-2\@tempdima\relax%
   \itshape}
  {\par\normalfont\hfill--\ \chapquote@author\hspace*{\@tempdima}\par\bigskip}
\makeatother

\newcommand{\cmd}[1]{\textcolor{blue}{\texttt{\$ #1}}}

%%%%%%%%%%%%%%%%%%%%%%%%%%%%%%%%%%%%%%%%%%%%%%%%%%%
% First page of book which contains 'stuff' like: %
%  - Book title, subtitle                         %
%  - Book author name                             %
%%%%%%%%%%%%%%%%%%%%%%%%%%%%%%%%%%%%%%%%%%%%%%%%%%%

% Book's title and subtitle
\title{\Huge \textbf{Computer Skills}  \footnote{Notebook of Yiqi Xinzhu} \\ \huge Glory to God the Creator of the Universe}
% Author
\author{\textsc{Qiang Chen}\thanks{\url{c275633094@gmail.com}}}


\begin{document}
\begin{CJK}{UTF8}{gbsn}
\frontmatter
\maketitle

%%%%%%%%%%%%%%%%%%%%%%%%%%%%%%%%%%%%%%%%%%%%%%%%%%%%%%%%%%%%%%%
% Add a dedication paragraph to dedicate your book to someone %
%%%%%%%%%%%%%%%%%%%%%%%%%%%%%%%%%%%%%%%%%%%%%%%%%%%%%%%%%%%%%%%
\begin{dedication}
Dedicated to God the heavenly father who created all, and the Lord Jesus Christ my savior.
\end{dedication}

%%%%%%%%%%%%%%%%%%%%%%%%%%%%%%%%%%%%%%%%%%%%%%%%%%%%%%%%%%%%%%%%%%%%%%%%
% Auto-generated table of contents, list of figures and list of tables %
%%%%%%%%%%%%%%%%%%%%%%%%%%%%%%%%%%%%%%%%%%%%%%%%%%%%%%%%%%%%%%%%%%%%%%%%
\tableofcontents
\listoffigures
\listoftables

\mainmatter

%%%%%%%%%%%
% Preface %
%%%%%%%%%%%
\chapter*{Preface}
The website\footnote{\url{https://github.com/amberj/latex-book-template}} for this file contains:

%%%%%%%%%%%%%%%%%%%%%%%%%%%%%%%%%%%%
% Give credit where credit is due. %
% Say thanks!                      %
%%%%%%%%%%%%%%%%%%%%%%%%%%%%%%%%%%%%
\section*{Acknowledgements}
\begin{itemize}
\item A special word of thanks goes to Jesus Christ.
\item I'll also like to thank my parents and my brother.

\end{itemize}
\mbox{}\\
%\mbox{}\\
\noindent Amber Jain \\
\noindent \url{http://amberj.devio.us/}

%%%%%%%%%%%%%%%%
% NEW CHAPTER! %
%%%%%%%%%%%%%%%%

\chapter{电脑技术}
\section{视频合成}

\cmd{mkvmerge\quad -o\quad output.mp4\quad input1.mp4 $\backslash+$ input2.mp4 $\backslash+$ input3.mp4}

\section{音频}
\subsection{微信音频提取}
Google Play : Voice Exporter for wechat\\
\subsection{音频拼接}
\cmd{mp3wrap output.mp3 *.mp3}

\chapter{PDF}
\section{Crack password}

PDF user password is authority in even reading, ower password is authority in editing.

\cmd{sudo apt install pdfcrack}

\cmd{pdfcrack} to see funtions options, \\

\cmd{pdfcrack -f testpdf.pdf}\\

\cmd{pdfcrack -f testpdf.pdf -o}\\
Here, -o means ower password 

\section{Edit index}

\cmd{./jpdfbookmarks} start app

\section{Edit text etc}

masterpdfeditor4


\chapter{Linux}

\section{command}

\cmd{ls -sh},  list contents in size, in human readable unit
\cmd{ls -i}, list contents in size, machine readable unit

\section{install}
\cmd{sudo apt install appname}

\cmd{sudo apt remove appname}

\cmd{sudo apt purge appname}
\section{unzip}

\begin{table}[ht]
\caption{unzip} % title of Table
\centering % used for centering table
\begin{tabular}{c c c c}
% centered columns (4 columns)
\hline\hline %inserts double horizontal lines
File name & untar & compile & notes \\ [0.5ex]
% inserts table
%heading
\hline % inserts single horizontal line
*.tgz & tar xvf  &  & 970 \\
5 & 45 & 300 & 556 \\ [1ex] % [1ex] adds vertical space
\hline %inserts single line
\end{tabular}
\label{table:nonlin} % is used to refer this table in the text
\end{table}


%\begin{CJK}{UTF8}{gkai}
%\noindent 间隔
%\end{CJK}

\chapter{CAMK}
\section{psk}
\subsection{Laptop}
Laptop to connect to psk: \cmd{ssh chen@ssh.camk.edu.pl}
\subsection{Office Desktop}
Office computer connection: \cmd{ssh chen@chen}\\
The first chen is my camk account, the second chen is my computer name. 

\paragraph{Cluster}

Chuck: SLURM  allocate jobs
\\
submit the job: \cmd{sbatch example.sh}\\
monitor jobs: \cmd{squeue/sacct/scontrol}
\\
PD pending, R running, CD completed, CA canceled, F failed\\
kill the job with \cmd{scancel}
\\
MPI jobs
\\

\paragraph{Sync Laptop and Desktop}
\cmd{rsync -azP ~/localdir chen@ssh.camk.edu.pl:Document} this will sync localdir to ~/Document of Desktop
\\
\cmd{rsync -azP chen@ssh.camk.edu.pl:~/Documents /home/jesuslovesme}, this will sync Documents dir from Desktop to laptop

\section{Jupyter}
In Alex's lecture in 2018 spring, use \cmd{ssh} go to \cmd{/work/alex/lect} and copy the file, use command to start the project: \cmd{jupyter notebook}\\

\end{CJK}
\end{document}

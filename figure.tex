\chapter{Figure}

\section{Convert format}

\begin{itemize}
\item \cmd{convert  -resize 50\% source.png dest.jpg} reduce figure size
\item flag \blue{-sDEVICE} declare the png format
\item flag \blue{-r} is the dpi
\item \cmd{gs -dNOPAUSE -sDEVICE=jpeg -r144 -sOutputFile=p\%03d.jpg file.pdf} convert pdf to jpeg
\item \cmd{gs -sDEVICE=pngalpha -o output.png input.pdf}
\item 
\begin{tabular}{ll}
\hline
pngalpha & alpha background\\
png16m & colorful background\\
png256 & colorful with 256 bit\\
png16 & colorful low quality\\
pnggray & gray\\
\hline
\end{tabular}
\item \cmd{gs -sDEVICE=pngalpha -dFirstPage=10 -dLastPage=20 -o out-\%03d.png -r500 input.pdf} convert PDF from 10 to 20 pages to png
\item \cmd{gs -dNOPAUSE -dBATCH -sDEVICE=pdfwrite -dCompatibilityLevel=1.4 -dPDFSETTINGS=/ebook -sOutputFile=output.pdf input.pdf} compress pdf size. PDFSETTINGS options are dpi in decreasing sequence as default, prepress, printer, ebook and screen.
\item \cmd{tiff2pdf -o out.pdf in.tif} convert tif figure into PDF
\item \cmd{mogrify -format jpg *.bmp} convert bmp to jpg
\item \cmd{mogrify -format png *.jpg} convert jpg to png
\item \cmd{mogrify -resize 320x240 *.jpg} resize
\end{itemize}

\subsection{Convert PDF to YouTube ratio PNG and crop}

\begin{itemize}
\item\cmd{convert -density 150 book.pdf[0] -quality 90 out.png} convert pdf page $1$ to png format
\item\cmd{identify fig.png} check the size of the figure
\item\cmd{convert fig.png -crop 1600x900+0+150 out.png} crop YouTube ratio
\item\cmd{convert fig.png -crop 1920x1080+0+150 out.png} crop YouTube ratio
\end{itemize}

\section{Crop}

\begin{itemize}
\item \cmd{convert input.png -trim output.png} trim all the margins
\item \cmd{convert input.png -trim info:} print the margin (edge) infomation
\item \cmd{convert test.png -trim  -format '\%[fx:w]x\%[fx:h]+\%[fx:page.x]+\%[fx:page.y]' info:} print trimed size and starting coordinate
\item \cmd{convert test.png -trim  -format '\%[fx:w+20]x\%[fx:h+20]+\%[fx:page.x-10]+\%[fx:page.y-10]' info:} print a expanded trimmed figure size
\item \cmd{convert fig.png -crop 1600x900+0+150 out.png}  rm margin wxh+x0+y0
\item \cmd{display example.jpg} render image on the screen, left click > "Transform" > crop.
\end{itemize}

\section{Append}
\begin{itemize}
\item \cmd{convert image1.png image2.png image3.png -append result.png} vertical append
\item \cmd{convert image1.png image2.png image3.png +append result.png} horizontal append
\end{itemize}


\section{Rotate image}
\begin{itemize}
\item \cmd{convert input.jpg -rotate 90 output.jpg} rotate image 90 degree
\end{itemize}


\section{Change DPI}

我有点迷糊DPI 与resolution的含义关系。现更改resolution吧

\begin{itemize}
\item \cmd{identify -format '\%x,\%y\n' imagefile} resolution in ppi (pixels per inch)
\item \cmd{gimp imagefile, alt+enter} check resolution by GIMP
\item \cmd{identify -verbose fig_in} 查看resolution
\item \cmd{convert fig_in -density 610 fig_out} 改变resolution
\item \cmd{convert -units PixelsPerInch fig_in -density 610 fig_out}
\item \red{爲什麼改了DPI圖片大小不變?}
\end{itemize}

\section{Resize}

\begin{itemize}
\item \cmd{convert -resize 20\% fig_in fig_out} reduce to 20 percent
\end{itemize}


\section{Anomation}

\cmd{convert -delay 20 -loop 0 input*.png out.gif}
\begin{itemize}
\item \verb|-delay {time}| unit in 1/100th of a second
\item \verb|-loop {number}| play how many times. But 0 means non stop. 
\end{itemize}

\section{Background}
\begin{enumerate}
\item\cmd{convert image1.jpg -fuzz 20\%\% -transparent White image2.png} 
\item\cmd{convert image1.png -threshold 10\%\% image2.png}
\item\cmd{backgroundremover -i "/path/to/file.jpg" -o "out.png"}
\item\cmd{backgroundremover -i "/path/to/video.mp4" -tg -o "output.gif"} video source
\end{enumerate}

\section{Screeshot}
\begin{enumerate}
\item\cmd{screengrab -a} screenshot active window
\item\cmd{screengrab -r} screenshot region
\item\cmd{sleep 2 && screengrab -a} launch screengrab after 2 seconds
\end{enumerate}

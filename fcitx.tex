\chapter{Fcitx}

\section{小技巧}

\begin{itemize}
\item \cmd{; + aphabet} print digital emotions: a - amazing, b- bear, bye,  e - effort, l-love,
\item \cmd{llll} 中文大号圆圈“一九九○”(\blue{Cjk} 无法显示)
\item \cmd{lklu} 中文日期圆圈号 “二○二○”(\blue{Cjk} 下可使用这个)
\item \cmd{Ctrl+.} 中文下为中英标点切换,英文下为符号表情。
\item 输入框选字翻頁:-/=
\item\cmd{ctrl+;} clipboard history
\end{itemize}

\section{词库}
\subsection{编辑或查询自定义词库}
\cmd{sudo apt install fcitx-tools} 安装编辑字库的工具

fcitx 自带的五笔词库: /usr/share/fcitx/table/wbx.mb 或~/.config/fcitx/table/里。转换词库为可编辑文本:

\cmd{mb2txt wbx.mb >> wbx.txt}

上面生成的 wbx.txt 是纯文本文件,修改完后,用下面的命令转换成二进制词库:

\cmd{txt2mb wbx.txt wbx.mb}

\cmd{fcitx -r} 重启输入法

\subsection{自动造词和删减}
\begin{itemize}
\item \cmd{ctrl +8} 选最近的字造词。方法二,先单字打字,再连续打,即造词成功
\item \cmd{ctrl +7} 在有输入框的时候,可以从词库中删词,\red{不会用}
\item \cmd{ctrl +6} 修改频率,\red{不会用}
\end{itemize}

\section{五笔}
\begin{multicols}{3}
\begin{itemize}
\itemsep0em
\item 凸 hgmg
\item 
\end{itemize}
\end{multicols}

\section{二笔}

\cmd{sudo apt install fcitx-table-erbi} 安装青松二笔(二笔标版)。青松二笔、纯净二笔属于原二笔。超强二笔采用了末笔。青松二笔、纯净二笔和超强二笔均采用原版二笔的键盘图。

\cmd{sudo dpkg -i ~/Apps/fcitx-table-cqlb.deb} 安装离线的超强二笔


\begin{tabular}{cccccccccc}
\multicolumn{10}{c}{青松二笔一级简码} \\
\hline
起Q&	为W&	而E&	人R&	他T&	一Y&	大U&	有I&	我O&	平P\\
安A&	是S&	的D&	分F&	个G&	和H&	就J&	可K&	了L&	*\\
在Z&	学X&	成C&	这V&	不B&	你N&	们M&	*&	*&	*\\
\hline
\multicolumn{10}{c}{超强二笔(或超强音形)一级简码} \\
\hline
起Q&	为W&	而E&	人R&	他T&	一Y&	以U&	有I&	我O&	平P\\
安A&	是S&	的D&	分F&	个G&	和H&	就J&	可K&	了L&	*\\
在Z&	学X&	成C&	这V&	不B&	你N&	们M&	*&	*&	*\\
\hline
\end{tabular}

\subsection{超强二笔打字规则}
\begin{itemize}
\item 独体字:拼音首字母 + 前两笔 + 末笔\\
如: 雨 YJV = Y(首音)+ J(一丨)+ V(丶)
\item 合体字 后半是合体结构:拼音首字母 + 前半前两笔 + 后半首部前两笔 + 后半次部前两笔\\
如: 撕 SUJE = S(首音)+ U(扌)+ J(一丨)+ E(ノノ)
\item 合体字 后半是独体结构:拼音首字母 + 前半前两笔 + 后半前两笔 + 后半末笔\\
如: 铺 PZJV = P(首音)+ Z(钅)+ J(一丨)+ V(丶)
\item 字根:Z钅X木C氵V土B艹S日D月F亻L口U扌\\
口诀:金木水土草,日月人口手。\\
注:金=钅 水=氵 曰=日 人=亻 手=扌\\
字根整体取码,不能拆分为笔画。\\
如字根有其它笔画穿过,则不再视为字根。\\
如:土 TV = T(首音) + V(字根)\\
如:教 JJQV = J(首音)+ J(二)+ Q(ノ一)+ D(丶)
\item 打词:\\
二字词:取每字前两码。\\
如: 教程 JJCQ\\
三字词:取第一字前两码和后两字第一码。\\
如: 输入法 S;RF\\
四字词:取每字第一码。\\
如:超强二笔 CQEB\\
多字词:取前三字和末字第一码。\\
如: 中华人民共和国 ZHRG
\item 全形输入与拼音输入
全形可以输入偏旁\\
不会读的字可选择全形方式,方法是“i+单字全形”。\\
如:瘿 IYG,\\
如:\red{首 SIWZ,IIWZ} 难道“首”字是上下结构的合体字?\\
不会写的字可选择拼音方式,方法是“i+单字拼音”。\\
如:睿 IRUI
\end{itemize}

\section{latex}
\cmd{sudo apt install fcitx-table-latex} 用于在非\LaTeX{}环境下输入各种字符
